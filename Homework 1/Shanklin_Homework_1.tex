\documentclass{article}
\usepackage[utf8]{inputenc}
\usepackage{siunitx}

\title{AA 100 Homework 1}
\author{Oliver Shanklin}
\date{1 February 2019}

\begin{document}

\maketitle

\section*{Problem 1}
At an average of 65 miles per hour, it would take about 2.707 hours to travel 176 miles.
\begin{equation}
    \frac{176 mi}{65 mph} = 2.707 hrs
\end{equation}

\section*{Problem 2}
Light travels at 300,000 km/s
\subsection*{a)}
Light traveling for 4 hours at 300,000 km/s. Convert to km/4hr:
\begin{equation}
    300,000km/hr * 60s * 60min * 4hr = 4.32x10^9
\end{equation}
\begin{equation}
    \frac{4.32x10^9}{150,000,000} = 28.8 AU
\end{equation}
So the radio waves traveled 28.8 AU in 4 hours.
\subsection*{b)}
Light traveling for 17 hours at 300,000 km/s. Convert to km/17hr:
\begin{equation}
    300,000km/hr * 60s * 60min * 17hr = 1.512x10^10
\end{equation}
\begin{equation}
    \frac{1.512x10^10}{150,000,000} = 100.8 AU
\end{equation}

\section*{Problem 3}
The distance from Earth to Mercury is 0.62 AU.
\subsection*{a)}
To send a radio signal to Mercury from Earth, it would take 310 seconds. The speed of light is 0.002 AU/s.
\begin{equation}
    \frac{.62 AU}{.002 AU/s} = 310 seconds
\end{equation}
\subsection*{b)}
The signal would not make it to the rover in time. Since it only has 120 seconds to respond to the cliff, the signal would reach the rover 190 seconds after the rover fell off the cliff.
\section*{Problem 4}
The ecliptic is the plane of the path of the Sun in the sky. It signifies what Zodiac sign the Sun is in currently.

\section*{Problem 5}
Occam's Razor is the idea that simpler solutions are more likely to be correct than complex answers.

\section*{Problem 6}
Circumpolar Constellations are visible throughout the year in Northern or Southern Hemispheres. Like, Ursa Major and Ursa Minor in the Northern Hemisphere.

\section*{Problem 7}
The sky from Fort Collins rotates from East to West at about \ang{40}.
\subsection*{a)}
Near the North Pole, the sky still rotates clockwise like in Fort Collins, but if you look directly up, that part of the sky will effectively never move, just rotate.
\subsection*{b)}
Near the Equator, the sky would have a perfect Sunrise and Sunset. Basically, a point directly to the East would follow a perfect line in the sky over top and then to the West. Nothing at the Equator would rotate around you like in Fort Collins or the North Pole.
\subsection*{c)}
Near the South Pole, the sky would rotate in the reverse direction from the North Pole, so counter-clockwise. Everything would rotate around you.

\end{document}

